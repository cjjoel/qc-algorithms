\documentclass[12pt]{article}
\usepackage{tikz}
\usepackage{amsmath}
\usetikzlibrary{quantikz}
\author{}
\date{}
\title{Superdense Coding}
\usepackage{braket}
\begin{document}
\maketitle
\section{Quantum Circuit}
  \begin{quantikz}
    \lstick{$q_0$} & \gate{H} & \ctrl{1}\slice{$\ket{\psi_1}$} & \gate{X} & \gate{Z}\slice{$\ket{\psi_2}$} & \ctrl{1}  & \gate{H}\slice{$\ket{\psi_3}$} & \meter{} & \qw & \rstick{$\ket{1}$}\\
    \lstick{$q_1$} & \qw & \targ{} & \qw & \qw & \targ{} & \qw & \meter{} & \qw & \rstick{$\ket{1}$}
  \end{quantikz}
  \section{State Vector}
  At first 2 qubits are entangled to form a bell state $\ket{\phi^+}$\\ 
  \begin{equation*}
    \begin{split}
      \ket{\psi_1} & = \ket{0} \otimes \ket{0}\\
      & =\frac{1}{\sqrt{2}}(\ket{0} + \ket{1}) \otimes \ket{0}\\
      & = \frac{1}{\sqrt{2}}(\ket{00} + \ket{10})\\
      & = \frac{1}{\sqrt{2}}(\ket{00} + \ket{11})\\
      & = \ket{\Phi^+}
    \end{split}
  \end{equation*}\\
  There are in total 4 bell states, transformation between them is carried out with quantum gates\\
  \begin{equation*}
    \begin{split}
      I\ket{\Phi^+} = \frac{1}{\sqrt{2}}(\ket{00} + \ket{11}) \ket{\Phi^+}\\
      X\ket{\Phi^+} = \frac{1}{\sqrt{2}}(\ket{00} - \ket{11}) \ket{\Phi^-}\\
      Z\ket{\Phi^+} = \frac{1}{\sqrt{2}}(\ket{01} + \ket{10}) \ket{\Psi^+}\\
      ZX\ket{\Phi^+} = \frac{1}{\sqrt{2}}(\ket{01} - \ket{10}) \ket{\Psi^-}
    \end{split}
  \end{equation*}\\
  Each of these bell states represent a 2 bit binary when measured under bell basis\\
  \begin{equation*}
    \begin{split}
      \ket{\Phi^+} \xrightarrow{} 00 = B_0\\
      \ket{\Psi^+} \xrightarrow{} 01 = B_1\\
      \ket{\Phi^-} \xrightarrow{} 10 = B_2\\
      \ket{\Psi^-} \xrightarrow{} 11 = B_3
    \end{split}
  \end{equation*}\\
  Here, the data to be transmitted is 11. Thus X and Z gates are used\\
  \begin{equation*}
    \begin{split}
      \ket{\psi_2} & = \ket{\Phi^+}\\
      & = \frac{1}{\sqrt{2}}(\ket{00} + \ket{11})\\
      & = \frac{1}{\sqrt{2}}(\ket{10} + \ket{01})\\
      & = \frac{1}{\sqrt{2}}(-\ket{10} + \ket{01})\\
      & = \ket{\Psi^-}
    \end{split}
  \end{equation*}\\
  Final step is to do bell basis measurement on the state vector and measure it to decode the sent data\\
  \begin{equation*}
    \begin{split}
      \ket{\psi_3} & = \ket{\Psi^-}\\
      & = \frac{1}{\sqrt{2}}(\ket{01} - \ket{10})\\
      & = \frac{1}{\sqrt{2}}(\ket{01} - \ket{11})\\
      & = \frac{1}{\sqrt{2}} \left( \left( \frac{\ket{0} + \ket{1}}{\sqrt{2}} \otimes \ket{1} \right) - \left( \frac{\ket{0} - \ket{1}}{\sqrt{2}} \otimes \ket{1} \right) \right)\\
      & = \frac{1}{2}(\ket{01} + \ket{11} - \ket{01} + \ket{11})\\
      & = \ket{11}
    \end{split}
  \end{equation*}\\
\end{document}
